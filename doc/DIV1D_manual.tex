\documentclass[amsmath,amssymb,a4]{revtex4}
\usepackage{graphicx}
\begin{document}

\title[DIV1D manual]{DIV1D manual: a 1D code for divertor plasma simulation}

\author{E. Westerhof}

\address{DIFFER -- Dutch Institute for Fundamental Energy Research, PO Box 6336, 5600HH Eindhoven, The Netherlands, www.differ.nl}

\email[E-mail: ]{e.westerhof@differ.nl}
\vspace{10pt}
\date\today

\begin{abstract}
Abstract goes here.
\end{abstract}


\maketitle

\section{Introduction}

This manual describes the {\tt DIV1D} code for the quick simulation of the 1D plasma and neutral behaviour along a flux tube in a tokamak divertor from the X-point up to the target. The code is inspired by similar works described in the papers by Nakazawa et al.~\cite{nakazawa2000} and Dudson et al.~\cite{dudson2019, SD1D}. In fact, the numerical implementation borrows heavily from the methods used in the SD1D code.


\section{Equations}

The equations solved are: the particle balance equation, the plasma momentum balance, the plasma energy balance, and an equation for the evolution of the neutral particle density. As in Nakazawa et al.~\cite{nakazawa2000} the neutral momentum and energy are ignored.

\noindent The particle balance is given by
\begin{equation}\label{particle_balance}
    {\partial n \over \partial t} = - {\partial \over \partial x} \Gamma_n + S_n,
\end{equation}
where $n$ is the plasma (electron) density, $\Gamma_n = n v_\parallel$ with $v_\parallel$ the parallel velocity is the convective particle flux (a possible effect of diffusion is ignored), and $S_n$ represent the sum of all particle sources and sinks

\noindent The momentum balance is given by
\begin{equation}\label{momentum_balance}
    {\partial n m v_\parallel \over \partial t} = - {\partial \over \partial x} \left( n m v_\parallel^2 + p \right) + S_{\rm mom},
\end{equation}
where $m$ is the mass of the dominant ion species (default Deuterium $m = 3.3436 \times 10^{-27}$~kg) , $p = 2 n e T$ is the total plasma pressure (we will use T in units eV such that the Boltzmann constant can be equated with the elementary charge $e$ to obtain the pressure in Pascal), and $S_{\rm mom}$ represent the sum of all momentum sources and sinks.

\noindent Ion and electron temperatures are considered equal such that only one energy balance needs to be solved. The internal energy ($3 n k T$) balance of the plasma is then given by
\begin{equation}\label{energy_balance}
    {\partial 3 n e T \over \partial t} = - {\partial \over \partial x} q_\parallel + v {\partial \over \partial x} p + Q,
\end{equation}
where the heat flux $q_\parallel$ is given by the equation
\begin{equation}\label{heat_flux}
    q_\parallel = 5 n e T v_\parallel - \kappa_\parallel {\partial \over \partial x} T,
\end{equation}
where the first term on the right hand side represents the total enthalpy flux and the second term represents the parallel heat conduction with (for T in eV) the parallel conductivity being given by $\kappa_\parallel = 2 \times 10^3 T^{5/2}$ J/eVms (see Chapter 4.10.1 of \cite{stangeby}). $Q$ represent the sum of all internal energy sources and sinks.

\noindent The neutral particle balance is given by
\begin{equation}\label{neutral_particle_balance}
    {\partial n_n \over \partial t} = {\partial \over \partial x} D {\partial \over \partial x} n_n - S_n,
\end{equation}
where $n_n$ is the neutral particle density and D is the neutral particle diffusion coefficient which is given by \cite{nakazawa2000}
\begin{equation}\label{neutral_diffusion_coefficient}
    D = { eT \over m \nu_{\rm cx} \sin^2\theta }
\end{equation}
where $\theta$ is the angle at which the magnetic field hits the target, and $\nu_{\rm cx} = n <\sigma_{\rm cx} v>$ is the charge exchange collision frequency for the neutrals and the average charge exchange rate $<\sigma_{\rm cx} v>$ is specified below.


\section{Boundary Conditions}

Each of these equations requires boundary conditions at the X-point $x=0$ and at the target $x=L$, where $L$ is the given length of the flux tube. At the X-point we use the following boundary conditions:
the plasma density at the X-point is given: $n(x=0) = n_{\rm X}$; the plasma momentum flux at the X-point assumed to be constant: the parallel heat flux at the X-point is given:  $q_\parallel(x=0) = q_{\parallel,\rm X}$; and finally, the neutral particle density is assumed to have zero gradient.

The boundary conditions at the target are given by the usual sheath boundary conditions assuming that density and temperature are constant across the sheath, while the plasma particle flux and momentum are given by the Bohm condition, $\Gamma_n(x=L) \ge n c_s$ and  where $c_s = \sqrt{2eT/m}$ is the plasma sound speed, and the the heat flux on the target is given by the sheath heat transmission factor $\gamma$ which must be specified at input:
\begin{equation}\label{sheath_heat_transmission}
    q_\parallel(x=L) = \gamma n e T c_s.
\end{equation}
The neutral particle flux coming from the target is determined by the recycling coefficient $R$ given at input, i.e. $\Gamma_{\rm neutral}(x=L) = - R \Gamma_n(x=L)$. Note that when $R=1$ the total number of particles should be conserved requiring a zero plasma inflow velocity at the target in case of a steady state solution.


\section{Sources and Sinks}

The various sources and sinks are determined mostly by atomic processes like charge exchange, ionization, excitation, and recombination. In addition a neutral particle loss term can be specified in terms of an average residence time for the neutral in the flux tube to account for cross field neutral particle transport losses. For the plasma density the sources and sinks are given by ionization and recombination, respectively:
\begin{equation}\label{particle_source}
    S_n = + n n_n <\sigma_{\rm ion} v> - n^2 <\sigma_{\rm rec} v>.
\end{equation}
Because the neutral particle momentum is neglected there is no momentum source, while the momentum sinks are induced by charge exchange and recombination
\begin{equation}\label{momentum_source}
    S_{\rm mom} = - n m v_\parallel \left(n_n <\sigma_{\rm cx} v> + n <\sigma_{\rm rec} v> \right).
\end{equation}
Similarly there are only heat sinks in the energy equation, with additional sinks coming from excitation (including ionization) and impurity radiation
\begin{equation}\label{energy_source}
    Q = - 1.5 n e T \left(n_n <\sigma_{\rm cx} v> + 2 n <\sigma_{\rm rec} v> \right) - n n_n <\sigma_{\rm exc} v> - n^2 \xi_Z L_Z(T)
\end{equation}
where $L_Z(T)$ is the radiative cooling rate of the impurity with concentration $\xi$. What is a sink for the plasma density is a source for the neutral particles and vice versa, such that
\begin{equation}\label{neutral_source}
    S_{neutral} = - S_n - {n_n \over \tau_n}
\end{equation}
where $\tau_n$ is a finite residence time of the neutral particles. The atomic rates for charge exchange, recombination, ionization, and excitation as used in the code as given in the next paragraph where also the impurity radiative cooling rates are be given.


\section{Atomic Rates}

A number of different options is available to calculate the atomic rates. For each of the rates a number of options is provided for different expressions that have been used in the literature. These include approximate formulas which are used amongts others in the SD1D code \cite{SD1D} of which the origin is however not always fully clear. The to our knowledge more accurate rates are obtained from the AMJUEL data base that is also being used for the EIRENE neutral particle Monte Carlo code and which can be found at the web site of the code \cite{EIRENE}. By default the rates from the AMJUEL data base are selected.


\subsection{Charge Exchange}

The charge exchange rate as implemented in SD1D originates from the work by Havlickova \cite{havlickova2013} and is given by \cite{SD1D}
\begin{equation}\label{charge_exchange_SD1D}
    <\sigma_{\rm cx} v> = \begin{cases} 1.0 \times 10^{-14} {\rm m}^3/{\rm s}             & \mbox{for } T \le 1 {\rm eV} \\
                                        1.0 \times 10^{-14} T^{1/3} {\rm m}^3/{\rm s} & \mbox{for } T >   1 {\rm eV}. \end{cases}
\end{equation}
This option is selected by setting {\tt case\_cx = "Havlickova"}.

Another expression that has been used for example in the work by Nakazawa et al. \cite{nakazawa2000}, comes from Table 3 of the report by Freeman and Jones \cite{freeman1974} in which case the charge exchange rate is given by the fit function
\begin{equation}\label{charge_exchange_FJ}
    <\sigma_{\rm cx} v> = 10^{-6} \exp\left( \sum_{i=0}^8 b_i (\ln T)^i \right)  {\rm m}^3/{\rm s}
\end{equation}
with fitting coefficients $b_i$ given as
\begin{small}\begin{verbatim}
   b0 -1.841757E+01  b1  5.282950E-01  b2 -2.200477E-01
   b3  9.750192E-02  b4 -1.749183E-02  b5  4.954296E-04
   b6  2.174910E-04  b7 -2.530206E-05  b8  8.230751E-07
\end{verbatim}\end{small}
This option is selected by setting {\tt case\_cx = "Freeman"}. The validity range of this fit is indicated as 1 to $10^5$~eV.

When using the AMJUEL option the charge exchange rate is calculated using the same fit function as above
\begin{equation}\label{charge_exchange_AMJUEL}
    <\sigma_{\rm cx} v> = 10^{-6} \exp\left( \sum_{i=0}^8 b_i (\ln T)^i \right)  {\rm m}^3/{\rm s}
\end{equation}
with the fitting constants $b_i$ as defined in section 2.1 reaction 0.1T of the AMJUEL document for the total charge exchange rate of Hydrogen~\cite{EIRENE}
\begin{small}\begin{verbatim}
   b0 -1.833882000000E+01  b1  2.368705000000E-01  b2 -1.469575000000E-02
   b3 -1.139850000000E-02  b4  6.379644000000E-04  b5  3.162724000000E-04
   b6 -6.681994000000E-05  b7  3.812123000000E-06  b8  8.652321000000E-09
\end{verbatim}\end{small}
Note that the factor $10^{-6}$ stems from the use of the units ${\rm cm}^3/{\rm s}$ for the reaction rates in AMJUEL.

Charge exchange rates for the Hydrogen isotopes like Deuterium and Tritium are obtained by the same expressions given above, but using a rescaled temperature multiplied with the factor $m_p/m_i$, i.e. the ratio of the proton mass over the mass of the Deuterium or Tritium ion, respectively.


\subsection{Ionization}

The ionization rate as implemented in SD1D again originates from the work by Havlickova \cite{havlickova2013}. It is slightly modified to remove the discontinuity at 20~eV and is given by \cite{SD1D}
\begin{equation}\label{ionization_SD1D}
    <\sigma_{\rm ion} v> = \begin{cases} 7.638 \times 10^{-21} {\rm m}^3/{\rm s}             & \mbox{for } T \le 1 {\rm eV} \\
                                        10^{-6.0d+0} T^{-2.987} 10^{-15.72 \exp(-\log_{10}T) + 1.603*exp(-\log_{10}^2T)} {\rm m}^3/{\rm s} & \mbox{for } 1 {\rm eV} < T \le 20 {\rm eV} \\
                                        5.875 \times 10^{-12} T^{-0.5151} 10^{-2.563/\log_{10}T} {\rm m}^3/{\rm s}. \end{cases}
\end{equation}
This option is selected by setting {\tt case\_ion = "Havlickova"}

Nakazawa et al.~\cite{nakazawa2000} in their 1D code use the ionization rate given in Table 3 of the report by Freeman and Jones. They provide a fit to the electron impact ionization given by
\begin{equation}\label{charge_exchange_FJ}
    <\sigma_{\rm ion} v> = 10^{-6} \exp\left( \sum_{i=0}^6 b_i (\ln T)^i \right)  {\rm m}^3/{\rm s}
\end{equation}
with fitting coefficients $b_i$ given by
\begin{small}\begin{verbatim}
   b0 -0.3173850E+02  b1  0.1143818E+02  b2 -0.3833998E+01
   b3  0.7046692E+00  b4 -0.7431486E-01  b5  0.4153749E-02
   b6 -0.9486967E-04
\end{verbatim}\end{small}
This option is selected by setting {\tt case\_ion = "Freeman"}

When using the AMJUEL option the effective ionization rate is calculated using the double fit function
\begin{equation}\label{ionization_AMJUEL}
    <\sigma_{\rm ion} v> = 10^{-6} \exp\left( \sum_{i=0}^8\sum_{j=0}^8 \alpha_{ij} (\ln \bar n)^j (\ln T)^i \right)  {\rm m}^3/{\rm s}
\end{equation}
where the density is normalized as $\bar n \equiv n / 10^{14}$ and the fitting coefficients $\alpha_{ij}$ are given in section 4.3 reaction 2.1.5 of the AMJUEL document for the case of the total ionization rate (including all excited states of the neutral hydrogen atoms)
\begin{small}\begin{verbatim}
        n-Index:     0                     1                     2
  T-Index:
        0   -3.248025330340D+01   -5.440669186583D-02    9.048888225109D-02
        1    1.425332391510D+01   -3.594347160760D-02   -2.014729121556D-02
        2   -6.632235026785D+00    9.255558353174D-02   -5.580210154625D-03
        3    2.059544135448D+00   -7.562462086943D-02    1.519595967433D-02
        4   -4.425370331410D-01    2.882634019199D-02   -7.285771485050D-03
        5    6.309381861496D-02   -5.788686535780D-03    1.507382955250D-03
        6   -5.620091829261D-03    6.329105568040D-04   -1.527777697951D-04
        7    2.812016578355D-04   -3.564132950345D-05    7.222726811078D-06
        8   -6.011143453374D-06    8.089651265488D-07   -1.186212683668D-07

        n-Index:     3                     4                     5
  T-Index:
        0   -4.054078993576D-02    8.976513750477D-03   -1.060334011186D-03
        1    1.039773615730D-02   -1.771792153042D-03    1.237467264294D-04
        2   -5.902218748238D-03    1.295609806553D-03   -1.056721622588D-04
        3    5.803498098354D-04   -3.527285012725D-04    3.201533740322D-05
        4    4.643389885987D-04    1.145700685235D-06    8.493662724988D-07
        5   -1.201550548662D-04    6.574487543511D-06   -9.678782818849D-07
        6    8.270124691336D-06    3.224101773605D-08    4.377402649057D-08
        7    1.433018694347D-07   -1.097431215601D-07    7.789031791949D-09
        8   -2.381080756307D-08    6.271173694534D-09   -5.483010244930D-10

        n-Index:     6                     7                     8
  T-Index:
        0    6.846238436472D-05   -2.242955329604D-06    2.890437688072D-08
        1   -3.130184159149D-06   -3.051994601527D-08    1.888148175469D-09
        2    4.646310029498D-06   -1.479612391848D-07    2.852251258320D-09
        3   -1.835196889733D-06    9.474014343303D-08   -2.342505583774D-09
        4   -1.001032516512D-08   -1.476839184318D-08    6.047700368169D-10
        5    5.176265845225D-08    1.291551676860D-09   -9.685157340473D-11
        6   -2.622921686955D-09   -2.259663431436D-10    1.161438990709D-11
        7   -4.197728680251D-10    3.032260338723D-11   -8.911076930014D-13
        8    3.064611702159D-11   -1.355903284487D-12    2.935080031599D-14
T1MIN =   0.10000D 00 EV
T1MAX =   2.00000D 04 EV
N2MIN =   1.00000D 08 1/CM3
N2MAX =   1.00000D 16 1/CM3
\end{verbatim}\end{small}


\subsection{Excitation and ionization energy losses}

Energy losses from ionization and due to excitation radiation are accounted for either as an effective energy loss constant per ionization (typically $E_{ion} = 30$~eV as in Nakazawa et al. \cite{nakazawa2000}) which can be specified at input er by an effective excitation rate which accounts for the temperature and density dependent total energy loss from ionization and excitation radiation. In the SD1D code the latter is given by a simple fit function as
\begin{equation}\label{excitation_SD1D}
    <\sigma_{\rm exc} v> = {4.90 \times 10^{-13} \over 0.28 + Y} \exp(-Y) \sqrt{Y (1.0 + Y)} {\rm eV} {\rm m}^3/{\rm s},
\end{equation}
where $Y = 10.2 / \max( 1, T)$. Added to this is the 13.6~eV of energy loss per ionization.

When using the AMJUEL option the effective excitation rate is calculated from a fit function of the same form as defined above for the ionization rate (\ref{ionization_AMJUEL}) with the coefficients $\alpha_{{\rm exc,} ij}$ as tabulated in section 10.2 for reaction 2.5.1 of the AMJUEL document for the case of the total energy loss rate associated with Hydrogen ionization and excitation radiation:
\begin{small}\begin{verbatim}
        n-Index:     0                     1                     2
  T-Index:
        0   -2.497580168306D+01    1.081653961822D-03   -7.358936044605D-04
        1    1.004448839974D+01   -3.189474633369D-03    2.510128351932D-03
        2   -4.867952931298D+00   -5.852267850690D-03    2.867458651322D-03
        3    1.689422238067D+00    7.744372210287D-03   -3.087364236497D-03
        4   -4.103532320100D-01   -3.622291213236D-03    1.327415215304D-03
        5    6.469718387357D-02    8.268567898126D-04   -2.830939623802D-04
        6   -6.215861314764D-03   -9.836595524255D-05    3.017296919092D-05
        7    3.289809895460D-04    5.845697922558D-06   -1.479323780613D-06
        8   -7.335808238917D-06   -1.367574486885D-07    2.423236476442D-08

        n-Index:     3                     4                     5
  T-Index:
        0    4.122398646951D-04   -1.408153300988D-04    2.469730836220D-05
        1   -7.707040988954D-04    1.031309578578D-04   -3.716939423005D-06
        2   -8.328668093987D-04    2.056134355492D-04   -3.301570807523D-05
        3    4.707676288420D-04   -5.508611815406D-05    7.305867762241D-06
        4   -1.424078519508D-04    3.307339563081D-06    5.256679519499D-09
        5    2.411848024960D-05    5.707984861100D-07   -1.016945693300D-07
        6   -1.474253805845D-06   -2.397868837417D-07    1.518743025531D-08
        7   -4.633029022577D-08    3.337390374041D-08   -1.770252084837D-09
        8    5.733871119707D-09   -1.512777532459D-09    8.733801272834D-11

        n-Index:     6                     7                     8
  T-Index:
        0   -2.212823709798D-06    9.648139704737D-08   -1.611904413846D-09
        1   -4.249704742353D-07    4.164960852522D-08   -9.893423877739D-10
        2    2.831739755462D-06   -1.164969298033D-07    1.785440278790D-09
        3   -6.000115718138D-07    2.045211951761D-08   -1.790312871690D-10
        4    7.597020291557D-10    1.799505288362D-09   -9.280890205774D-11
        5    3.517154874443D-09   -4.453195673947D-10    2.002478264932D-11
        6    4.149084521319D-10   -6.803200444549D-12   -1.151855939531D-12
        7   -5.289806153651D-11    3.864394776250D-12   -8.694978774411D-15
        8    7.196798841269D-13   -1.441033650378D-13    1.734769090475D-15
T1MIN =   0.10000D 00 EV
T1MAX =   2.00000D 04 EV
N2MIN =   1.00000D 08 1/CM3
N2MAX =   1.00000D 16 1/CM3
\end{verbatim}\end{small}


\subsection{Recombination}

In their 1D code Nakazawa et al.~\cite{nakazawa2000} use a combination of the radiative recombination rate from Gordeev et al. plus the three body recombination rate given by Hinnov et al., where the former is given by the expression~\cite{gordeev1977}
\begin{equation}\label{radiative_recombination}
    <\sigma_{\rm rad.rec} v> = 1.27 \times 10^{-19} {(13.6./T)^{1.5} \over (13.6/T) + 0.59} {\rm m}^3 / s,
\end{equation}
and the latter is expressed as~\cite{hinnov1962}
\begin{equation}\label{three_body_recombination}
    <\sigma_{\rm 3bodyrec} v> = 5.6 \times 10^{-39} \; T^{-4.5} n {\rm m}^3 / s.
\end{equation}
This model is selected by setting {\tt case\_rec = "Nakazawa"}.

Otherwise, the recombination rate is used from the AMJUEL data providing the total effective recombination rate including 3 body recombination using again a fit function (\ref{ionization_AMJUEL}) as defined above for the ionization rate, now with the coefficients $\alpha_{{\rm rec,} ij}$ as tabulated in section 4.4 reaction 2.1.8 of the AMJUEL document:
\begin{small}\begin{verbatim}
        n-Index:     0                     1                     2
  T-Index:
        0   -2.858858570847D+01    2.068671746773D-02   -7.868331504755D-03
        1   -7.676413320499D-01    1.278006032590D-02   -1.870326896978D-02
        2    2.823851790251D-03   -1.907812518731D-03    1.121251125171D-02
        3   -1.062884273731D-02   -1.010719783828D-02    4.208412930611D-03
        4    1.582701550903D-03    2.794099401979D-03   -2.024796037098D-03
        5   -1.938012790522D-04    2.148453735781D-04    3.393285358049D-05
        6    6.041794354114D-06   -1.421502819671D-04    6.143879076080D-05
        7    1.742316850715D-06    1.595051038326D-05   -7.858419208668D-06
        8   -1.384927774988D-07   -5.664673433879D-07    2.886857762387D-07

        n-Index:     3                     4                     5
  T-Index:
        0    3.843362133859D-03   -7.411492158905D-04    9.273687892997D-05
        1    3.828555048890D-03   -3.627770385335D-04    4.401007253801D-07
        2   -3.711328186517D-03    6.617485083301D-04   -6.860774445002D-05
        3   -1.005744410540D-03    1.013652422369D-04   -2.044691594727D-06
        4    6.250304936976D-04   -9.224891301052D-05    7.546853961575D-06
        5   -3.746423753955D-05    7.509176112468D-06   -8.688365258514D-07
        6   -1.232549226121D-05    1.394562183496D-06   -6.434833988001D-08
        7    1.774935420144D-06   -2.187584251561D-07    1.327090702659D-08
        8   -6.591743182569D-08    8.008790343319D-09   -4.805837071646D-10

        n-Index:     6                     7                     8
  T-Index:
        0   -7.063529824805D-06    3.026539277057D-07   -5.373940838104D-09
        1    1.932701779173D-06   -1.176872895577D-07    2.215851843121D-09
        2    4.508046989099D-06   -1.723423509284D-07    2.805361431741D-09
        3   -4.431181498017D-07    3.457903389784D-08   -7.374639775683D-10
        4   -3.682709551169D-07    1.035928615391D-08   -1.325312585168D-10
        5    7.144767938783D-08   -3.367897014044D-09    6.250111099227D-11
        6   -2.746804724917D-09    3.564291012995D-10   -8.551708197610D-12
        7   -1.386720240985D-10   -1.946206688519D-11    5.745422385081D-13
        8    6.459706573699D-12    5.510729582791D-13   -1.680871303639D-14
T1MIN =   0.10000D 00 EV
T1MAX =   2.00000D 04 EV
N2MIN =   1.00000D 08 1/CM3
N2MAX =   1.00000D 16 1/CM3
\end{verbatim}\end{small}


\subsection{Impurity Radiation Losses}

The impurity radiation losses are typically given in the form of a radiative cooling function $L_Z(T)$ where $Z$ stands for the impurity species under consideration. For a given cooling rate function, the energy losses from impurity radiation are then given by
\begin{equation}\label{impurity_radiation}
    Q_{\rm imp} = - n^2 \xi_Z L_Z(T),
\end{equation}
where $\xi_Z$ is the concentration of the impurity. Post et al.\cite{post1977} tabulate fit functions for the most relevant impurities in fusion plasmas using the general functional form of \cite{post1977}
\begin{equation}\label{cooling_rate}
    \log_{10} L_Z = \sum_{i=0}^5 \; A(i) (\log_{10} T_{\rm keV})^i [{\rm cm}^3 {\rm erg/s}
\end{equation}
where $T_{\rm keV}$ is the temperature in keV.

In the present code version the radiation cooling function of Carbon is implemented. The coefficients of the fit function are given in the following table for various temperature ranges
\begin{small}\begin{verbatim}
  TMIN   TMAX       A(0)          A(1)          A(2)          A(3)          A(4)          A(5)
     3     20   1.965300E+03  4.572039E+03  4.159590E+03  1.871560E+03  4.173889E+02  3.699382E+01
    20    200   7.467599E+01  4.549038E+02  8.372937E+02  7.402515E+02  3.147607E+02  5.164578E+01
   200   2000  -2.120151E+01 -3.668933E-01  7.295099E-01 -1.944827E-01 -1.263576E-01 -1.491027E-01
\end{verbatim}\end{small}
Note that in order to convert to units of [W m$^3$] 13 is to be subtracted from the coefficient A(0).

An alternative for the cooling rate of Carbon according to Post et al. is provided in an expression from Havlickova \cite{havlickova2013}
\begin{equation}
    L_C(T) = 2.0 \times 10^{-31} {(T/10)^3 \over 1 + (T/10)^{4.5}} [{\rm W m}^3]
\end{equation}
for $T$ in eV.

\section{Discretization and numerical implementation}

The equations are discretized on a nonequidistant grid as employed also in the SD1D code \cite{SD1D}. For  $N$ grid cells the boundaries $x_{{\rm cb},i}$ counting from $i = 0$ at the X-point on the left to $i =N$ at the target on the right are given by \cite{SD1D}
\begin{equation}\label{cell_boundaries}
   x_{{\rm cb},i} = L \left( {(2 - \delta_{x,{\rm min}}) i \over N} - {(1 - \delta_{x,{\rm min}}) i^2 \over N^2} \right)
\end{equation}
where $L$ is the total distance from X-point to the target and $\delta_{x,{\rm min}}$ is a parameter that sets the ration between the smallest grid cell at the target to the average grid cell size. The cell centres are defined as
\begin{equation}\label{cell_centres}
    x_i = {x_{{\rm cb},i} + x_{{\rm cb},i-1} \over 2}, \quad\quad\hbox{for $i = 1 \cdots N$}.
\end{equation}
The widths of the grid cells are given by
\begin{equation}\label{cell_width}
    \Delta x_{{\rm cb},i} = x_{{\rm cb},i} - x_{{\rm cb},i-1}, \quad\quad\hbox{for $i = 1 \cdots N$}.
\end{equation}
Similarly, the distance between cell centres defines
\begin{equation}\label{deltax}
    \Delta x_i = x_{i+1} - x_{i}, \quad\quad\hbox{for $i = 1 \cdots N-1$},
\end{equation}
while for the final cell centre the distance to a virtual mirror point beyond the target is used to obtain
\begin{equation}
    \Delta x_N = 2 ( L - x_N ).
\end{equation}
All variables are calculated on the cell centres, while fluxes are calculated on the cell boundaries.

The primary variables that are evolved in the code are the plasma density $n$, the plasma momentum $P \equiv n m v_\parallel$, the total internal energy $E \equiv 3 n k T$, and the neutral density $n_n$. For evaluation of the ODE solver the normalized variables are stacked in a vector $Y$ of length $4N$ as
\begin{eqnarray}
    Y_i        &\equiv {\displaystyle n_i \over n_{\rm norm} } &\quad\quad\hbox{for $i = 1 \cdots N$}, \nonumber \\ \nonumber \\
    Y_{N + i}  &\equiv {\displaystyle P_i \over n_{\rm norm} m c_{\rm norm} } &\quad\quad\hbox{for $i = 1 \cdots N$}, \nonumber \\ \label{solution_vector} \\
    Y_{2N + i} &\equiv {\displaystyle E_i \over 3 n_{\rm norm} k T_{\rm norm} } &\quad\quad\hbox{for $i = 1 \cdots N$},  \nonumber\\ \nonumber \\
    Y_{3N + i} &\equiv {\displaystyle n_n \over n_{\rm norm} } &\quad\quad\hbox{for $i = 1 \cdots N$}, \nonumber
\end{eqnarray}
where $c_{\rm norm} = \sqrt{2 k T_{\rm norm}/m}$ is the sound speed at the normalizing temperature. Typically the normalizing density will be set equal to the initial density at the X-point $n_{\rm norm} = n_1$, while the normalizing temperature has the default value $T_{\rm norm} = 1$~eV.

The advected part of the fluxes are calculated according to the numerical scheme as used in SD1D \cite{SD1D}. First the parallel velocities on the cell boundaries are obtained from a simple average of the parallel velocities at the adjacent cell centres. The advected quantity is the obtained using a slope limiter and a upwind scheme scheme in case of supersonic flow. For subsonic flow a Lax flux is used to damp discontinuities. For details see Dudson et al. \cite{dudson2019} and references therein. Pressure gradients are discretized using a downwind scheme.


\section{Benchmark with 2 Point Model}

The 2 Point Model for an attached divertor plasma assuming purely conductive heat transport and no losses is discussed in Chapter 5.2 of~\cite{stangeby}). For a given parallel heat flux $q_{\parallel,\rm X}$ entering the divertor leg upstream, a given upstream plasma density $n_{\rm X}$, the length of the divertor leg $L$, a set of three equations is obtained from the energy balance, the sheath heat transmission, and pressure balance respectively~\cite{stangeby}:
\begin{equation}\label{2PM_energy_balance}
    T_{\rm X} = \left(T_{\rm L}^{7/2} + {7 \over 2} {q_{\parallel,\rm X} L \over \kappa_0}\right)^{2/7},
\end{equation}
\begin{equation}\label{2PM_sheath_heat_transmission}
    q_{\parallel,\rm X} = \gamma n e T_{\rm L} c_s,
\end{equation}
\begin{equation}\label{2PM_pressure_balance}
    2 n_{\rm L} T_{\rm L} = n_{\rm X} T_{\rm X}.
\end{equation}
with $\kappa_0 = 2000$~W/eV$^{7/2}$m. These equations can be used to solve for the three unknowns, upstream temperature $T_{\rm X}$, target temperature $T_{\rm L}$ and target density $n_{\rm L}$. The solution can be obtained iteratively starting from $T_{\rm L} = 0$. In the first iteration the solution for $T_{\rm X}$ then is
\begin{equation}\label{2PM_upstream_temperature}
    T_{\rm X} = \left({7 \over 2} {q_{\parallel,\rm X} L \over \kappa_0}\right)^{2/7}.
\end{equation}
Next an equation for $T_{\rm L}$ is obtained by substituting equation (\ref{2PM_pressure_balance}) in the sheath heat transmission (\ref{2PM_sheath_heat_transmission}):
\begin{equation}\label{2PM_target_temperature}
    T_{\rm L} = {m \over e} {2 q_{\parallel,\rm X}^2 \over \gamma^2 e^2 n_{\rm X}^2 T_{\rm X}^2}.
\end{equation}
Finally the pressure balance (\ref{2PM_pressure_balance}) then allows to obtain the target density as
\begin{equation}\label{2PM_target_density}
    n_{\rm L} = {\gamma^2 e^3 n_{\rm X}^3 T_{\rm X}^3 \over 4 m q_{\parallel,\rm X}^2}.
\end{equation}
In fact, these last three equations represent the simple 2 Point Model as obtained under the condition that $T_{\rm X} \gg T_{\rm L}$ \cite{stangeby}.

An extension of the simple 2 Point Model is obtained introducing the effects of a finite convective contribution to the parallel heat transport, as well as power and momentum losses (see Chapter 5.4 of~\cite{stangeby}). Assuming that the convective contribution is evenly distributed over the distance between the X-point and target, the solution to the energy balance equation \ref{2PM_energy_balance} is simply modified as
\begin{equation}\label{extended_2PM_energy_balance}
    T_{\rm X} = \left(T_{\rm L}^{7/2} + {7 \over 2} {q_{\parallel,\rm X} (1 - f_{\rm conv}) L \over \kappa_0}\right)^{2/7},
\end{equation}
introducing the factor $f_{\rm conv}$ representing the relative contribution from convection to the heat transport. When the power and momentum losses are not negligible yet sufficiently localized near the target, the energy balance solution for the upstream temperature (\ref{extended_2PM_energy_balance}) is not altered further, while the sheath heat transmission is simply modified by introducing the factor $(1-f_{\rm pwr})$ on the left hand side of equation (\ref{2PM_sheath_heat_transmission}) and the pressure balance is modified similarly by introducing the factor $(1-f_{\rm mom})$ on the right hand side of equation (\ref{2PM_pressure_balance}) to account for the power an momentum losses, respectively. As a result the equation for the target temperature (\ref{2PMF_target temperature}) is modified to
\begin{equation}\label{extended_2PM_target_temperature}
    T_{\rm L} = {m \over e} {2 q_{\parallel,\rm X}^2 (1-f_{\rm pwr})^2 \over \gamma^2 e^2 n_{\rm X}^2 T_{\rm X}^2 (1-f_{\rm mom})^2}.
\end{equation}
while the target density is modified to
\begin{equation}\label{extended_2PM_target_density}
    n_{\rm L} = {\gamma^2 e^3 n_{\rm X}^3 T_{\rm X}^3 (1-f_{\rm mom})^3 \over 4 m q_{\parallel,\rm X}^2 (1-f_{\rm pwr})^2}.
\end{equation}
The power and momentum loss fractions $f_{\rm pwr}$ and $f_{\rm mom}$ are defined as
\begin{equation}\label{power_loss_fraction}
    f_{\rm pwr} \equiv \int_{x=L}^0 Q {\rm d}x \Big/ q_{\parallel,\rm X},
\end{equation}
and
\begin{equation}\label{momentum_loss_fraction}
    f_{\rm mom} \equiv \int_{x=L}^0 S_{\rm mom} {\rm d}x \Big/ p_{\rm tot,X},
\end{equation}
respectively. Also the equations for the extended 2 Point Model (\ref{extended_2PM_energy_balance}-\ref{extended_2PM_target_density}) are easily solved by iteration when the convective fraction, and the power and momentum loss fractions are known.

For the case that the losses along the divertor leg and/or the convective heat flux are not negligible, Kotov and Reiter \cite{kotov2009} derived an exact 2 Point Model Formulation for the target temperature and density, which under the conditions of the present model (equal electron and ion temperature, sound speed at the sheath boundary, and constant magnetic field) is simplified to
\begin{equation}\label{2PMF_target temperature}
    T_{\rm L} = {8m \over e \gamma^2} {q_{\parallel,\rm X}^2 \left(1-f_{\rm pwr}\right)^2 \over p_{\rm tot,X}^2 \left(1-f_{\rm mom}\right)^2}
\end{equation}
\begin{equation}\label{2PMF_target density}
    n_{\rm L} = {\gamma^2 \over 32 m} {p_{\rm tot,X}^3 \left(1-f_{\rm mom}\right)^3 \over q_{\parallel,\rm X} \left(1-f_{\rm pwr}\right)^2}
\end{equation}
where $p_{\rm tot,X} = 2 n_{\rm X} e T_{\rm X} + n_{\rm X} m v_{\parallel\rm X}^2$ is the total upstream pressure. When the upstream parallel momentum is negligible, these equations are identical to the extended 2 Point Model equations (\ref{extended_2PM_target_temperature}) and (\ref{extended_2PM_target_density}) for the target temperature and density. These equations can serve as an accuracy check for the numerical code.


\subsection{Analytical Extended 2 Point Model}

When one would have simple closed analytical expressions for the power and momentum loss fraction, an instructive version of the extended 2 Point Model would be obtained. Such a model might provide further insight into conditions for divertor detachment and can be used for extremely fast calculations as needed in detachment control. Noting that both power losses and momentum losses typically increase with decreasing target temperature or increasing target density, we suggest the simple scaling
\begin{equation}\label{scaling_of_losses}
    f_{\rm pwr, mom} = exp( - \alpha_{\rm pwr, mom} T_{\rm L} / n_{\rm L} ),
\end{equation}
where the coefficients $\alpha_{\rm pwr, mom}$ must be chosen appropriately. Note that detachment (a decreasing target temperature) requires that $f_{\rm mom} < f_{\rm pwr}$, i.e. $\alpha_{\rm mom} < \alpha_{\rm pwr}$.

Some other remarks on the solutions of the 1D problem. Momentum dissipation requires a finite velocity, which only appears near the target where a finite particle source from ionization must be compensated by a convective particle flux towards the target. When recycling is incomplete or in case of efficient pumping in the divertor region, a significant particle flux must be present at the X-point already and there will be much more possibilities for momentum dissipation along the entire divertor leg. Impurity puffing will mostly effect power dissipation without significant momentum dissipation. In contrast, increasing the neutral density in the divertor by normal gass puff will increase the charge exchange rate and thereby increases momentum dissipation. Note that increasing the length of the divertor leg increases the X-point temperature. This must have a limit somewhere which sets a maximum distance from the X-point to the recombination front, when the divertor leg length becomes larger that this the detachment front must separate from the target. The question then arises about the stability of the detachment front position. How is this determined??


\section{Further analysis of the 1D divertor model: stationary solutions}

Important insight into the existence and stability of stationary solutions can be obtained by some further assumptions concerning the 1D model. In particular, ionization and recombination are often located in a relatively narrow zone. For stationary solutions of the equations, the pressure can then be assumed to be constant outside this narrow ionization and recombination zone and the parallel velocity and the convective heat flux can be neglected. The stationary solutions are then entirely specified by the stationary energy balance
\begin{equation}\label{stationary_energy_balance}
    {\partial \over \partial x} q_\parallel = Q.
\end{equation}
Using a transformation first introduced by Wim Goedheer [check citation] in the context of radiation losses from the edge of the confined plasma, allows to recast this equation into an integral equation for the change in the heat flux over the divertor leg or, in other words, an integral equation for the radiative energy losses in the divertor region. To this end Eq.~({\ref{stationary_energy_balance}) is multiplied by the parallel heat flux and integrated to obtain
\begin{equation}\label{flux_change}
    q_X^2 - q_L^2 = 2 \int_{T_L}^{T_X} \kappa_\parallel Q {\rm d}T.
\end{equation}
Next, the parallel heat conductivity is substituted as $\kappa_\parallel = \kappa_0 T^{5/2}$, the energy loss term is taken in the form of the radiative impurity losses (\ref{impurity_radiation}), and the constant pressure, $ p = 2 neT$, approximation is applied to obtain~\cite{lengyel1981,capes1992,hutchinson1994}
\begin{equation}\label{flux_change_2}
    q_X^2 - q_L^2 = {1\over2} \kappa_0 p^2 \xi_Z \int_{T_L}^{T_X} \sqrt{T} L_Z(T) {\rm d}T,
\end{equation}
where the impurity concentration $xi_Z$ has also been assumed to be constant over the radiative layer.

A number of conclusions can be drawn from this equation. Several authors apply it to quantify the maximum flux that can be radiated away in the divertor leg \cite{lengyel1981,lackner1993,kallenbach2013,siccinio2016}
\begin{equation}\label{maximum_radiated_flux}
    q_{\rm max} = \sqrt{{1\over2} \kappa_0 p^2 \xi_Z \int_0^\infty \sqrt{T} L_Z(T) {\rm d}T}.
\end{equation}
Hutchinson and Lipschultz use it to formulate the stability condition for a radiative front~\cite{hutchinson1994,lipschultz2016}:
\begin{equation}
    {{\rm d} \over {\rm d}x_f} (q_i - q_{\rm max}) \le 0,
\end{equation}
where $x_f$ is the front position and $q_i$ is the parallel energy flux entering the system. We apply this to analyse the stability of a detachment in our model where $q_i$ is fixed by the boundary condition at the X-point, while the pressure is determined by the equilibrium condition that $q_X = q_{\rm max}$. For a given X-point density then also the X-point temperature is determined, which through the heat conductivity equation sets the position of the radiation front (i.e. the position where the temperature is reached at which $L_Z(T)$ is maximum). Perturbing the front position in the direction of the X-point would lower the X-point temperature and, consequently the X-pint temperature and the pressure in the divertor. This will reduce the losses in the radiation front and push it back to its original position. Similarly a movement in the direction of the target will result in an increase of the losses again pushing the radiative layer back to its original position. As a result, given the boundary conditions in our model, the position of the radiation front for a fully detached case is always stable.

Capes et al. use this integral relation to the stability of attached or semi-detached divertor solutions for which $q_X$ is larger than $q_{\rm max}$~\cite{capes1992}. The residual energy flux to the target then determines the properties at the target according to the sheath boundary condition. Substituting the target heat flux (\ref{sheath_heat_transmission}) in Eq.~(\ref{flux_change_2}) and some algebraic manipulation results in a nonlinear relation between the target temperature $T_L$ to the conditions at the X-point:
\begin{equation}
    X(T_L) \equiv {2 m q_X^2 \over e \gamma^2 p^2} = T_L + {\kappa_0 m \xi_Z \over e \gamma^2} \int_0^\infty \sqrt{T} L_Z(T) {\rm d}T.
\end{equation}
Note that $X(T_L)$ is identical to the target temperature in case there are no radiative losses. For a given condition X at the X-point, there can be one, multiple, or no solution for $T_L$. The bifurcation points can be found by differentiating $X(T_L)$ with respect to $T_L$ and finding the zeros of the resulting relation, i.e.
\begin{equation}
    {{\rm d} X(T_L) \over {\rm d}T_L} = 1 - {\kappa_0 m \xi_Z \over e \gamma^2} \sqrt{T_L} L_Z(T_L).
\end{equation}
The impurity concentration $xi_Z$ above which the solutions for the stationary divertor solution are bifurcated is then determined by the maximum of $\sqrt{T} L_Z(T)$
\begin{equation}
    \xi_\star = {e \gamma^2 \over \kappa_0 m [\sqrt{T} L_Z(T)]_{\rm max}}.
\end{equation}
This explains the bifurcations that we have found in the stationary divertor solutions for Carbon impurity concentrations of 5\%. Note that in their paper Capes et al. considered only the electron fluid resulting in slightly different expressions, while also the parallel heat conductivity $\kappa_0$ and the heat transmission coefficient $\gamma$ differ from the resent work.

\section*{Acknowledgments}
\noindent This project was carried out with financial support from NWO. The work has been carried out within the framework of the EUROfusion Consortium and has received funding from the Euratom research and training programme 2014-2018 under grant agreement No 633053. The views and opinions expressed herein do not necessarily reflect those of the European Commission.

%\section*{References}
\begin{thebibliography}{99}
\bibitem{nakazawa2000}
Shinji Nakazawa, Noriyoshi Nakajima, Masao Okamoto and Nobuyoshi Ohyabu, Plasma Phys. Control. Fusion {\bf 42} (2000) 401.

\bibitem{dudson2019}
B.D. Dudson, J. Allen, T. Body, B. Chapman, C. Lau, L. Townley, D. Moulton, J. Harrison and B. Lipschultz, Plasma Phys. Control. Fusion {\bf 61} (2019) 065008.

\bibitem{SD1D}
Ben Dudson, {\it SD1D: 1D divertor model for detachment studies}, 19 December 2016.

\bibitem{stangeby}
Peter C. Stangeby, 2000, {\it The Plasma Boundary of Magnetic Fusion Devices}, Institute of Physics Publishing, Dirac House, Temple Back, Bristol BS1 6BE,
UK.

\bibitem{EIRENE}
{\tt http://www.eirene.de/}.

\bibitem{havlickova2013}
E. Havlickova, et al., Plasma Phys. Control. Fusion {\bf 55} (2013) 065004.

\bibitem{freeman1974}
R.L. Freeman, and E.M. Jones, {\it 'Atomic collision processes in plasma physics experiments'}, Culham Report CLM-R137

\bibitem{gordeev1977}
Yu.S. Gordeev, A.N. Zinov'ev, and M.P. Petrov, J. Exp. Theor. Physics {\bf 25} (1977) 204.

\bibitem{post1977}
D.E. Post, et al., Atomic Data and Nuclear Data Tables {\bf 20} (1977) 397.

\bibitem{hinnov1962}
E. Hinnov,. and J.G. Hirschberg, Phys. Review {\bf 125} (1962) 795.

\bibitem{kotov2009}
V. Kotov and D. reiter, Plasma Phys. Control. Fusion {\bf 51} (2009) 115002.

\bibitem{lengyel1981}
L.L. Lengyel, {\it 'Analysis of radiation Plasma Boundary Layers'}, Report IPP 1/191 (1981).

\bibitem{capes1992}
H. Capes, Ph. Ghendrih, and A. Samain, Phys. Fluids B {\bf 4} (1992) 1287.

\bibitem{hutchinson1994}
I.H. Hutchinson, Nucl. Fusion {bf 34} (1994) 1337.

\bibitem{lackner1993}
K. Lackner, and R. Schneider, Fusion Eng. Design {\bf 22} (1993) 107.

\bibitem{kallenbach2013}
A. Kallenbach, et al., Plasma Phys. Control. Fusion {\bf 55} (2013) 124041.

\bibitem{siccinio2016}
M. Siccinio, et al., Plasma Phys. Control. Fusion {\bf 58} (2016) 125011.

\bibitem{lipschultz2016}
B. Lipschultz, F.I. Parra, and I.H. Hutchinson, Nucl. Fusion {\bf 56} (2016) 056007.

\end{thebibliography}

\end{document}

\begin{eqnarray}
    &&2n_t T_t = n_u T_u ( 1 - f_{\rm mom}) \nonumber \\
    &&T_u^{7/2} = T_t^{7/2} + {\displaystyle {7 \over 2} {q_\parallel (1 - f_{\rm conv}) L \over \kappa_{0e}} } \nonumber \\
    &&q_\parallel (1 - f_{\rm pwr}) = \gamma n_t k_B T_t c_{st} \nonumber
\end{eqnarray}

